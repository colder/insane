\documentclass[a4paper]{article}
\usepackage[pdftex]{graphicx}
\usepackage{float}
\usepackage{hyperref}
\usepackage{caption}
\usepackage{placeins}
\usepackage{algorithm}
\usepackage{algorithmicx}
\usepackage{algpseudocode}
\usepackage{listings}
\usepackage{amssymb}
\floatstyle{ruled}
\newfloat{listing}{hbtH}{lop}
\floatname{listing}{Listing}
\renewcommand{\topfraction}{0.85}   % This sets the percentage for how much floats get from the ‘top’ of the text of a page
\renewcommand{\textfraction}{0}   % This sets a similar percentage for how much of a page needs to be text before no more floats can be placed on that page
\renewcommand{\floatpagefraction}{0.80} % This sets how much of the page must be taken by a float before that page can be ‘all’ floats

\newcommand{\subtypeeq}{\sqsubseteq}
\newcommand{\subtype}{\sqsubset}

\author{Etienne Kneuss\\
\texttt{etienne.kneuss@epfl.ch}
}
\title{Insane: Interprocedural Static Analysis Engine for Scala}
%\bibiographystyle{unsrt}
\begin{document}
\maketitle
%\begin{abstract}
%  This report presents the work that was done during the winter semester
%  2011-2012. It describes the various improvements and refinements that
%  were applied to the original analysis.
%\end{abstract}
\section{Introduction}
Insane is a pointer and effect analysis for Scala applications. It is meant to
compute and combine modular effects, with various degrees of precision. It was
specifically designed to address higher-order functions -- or equivalently
precise handling of dynamic dispatch. Insane is composed of a combination of
two analyses:

\begin{enumerate}
    \item a precise, intra-procedural but flow-sensitive \textbf{type analysis}
    \item a modular, inter-procedural pointer and \textbf{effect analysis}
\end{enumerate}

\section{Type Analysis}
\subsection{Overview}
Object oriented languages such as Scala implement \emph{dynamic dispatch}: the
target of a method call is only determined at runtime, based on the actual
runtime type of the receiver. This feature is essential in object oriented
languages as it allows subtype polymorphism. Obtaining precise information on
the possible runtime types of those receivers allows us to construct a
precise call-graph.

Type analysis is responsible to compute an over approximation of the set of
actual concrete types a variable could hold at runtime. Given a variable
\verb=v : T=, we know that its static type \verb=T= already provides a valid
bound on the set of possible concrete types, but this information is not
precise enough in practice. This analysis is designed to improve this static
bound by tracking how allocated objects flow in a procedure and derive more
precise type constraints. We use abstract interpretation using sets of type
constraints as abstract values.

\subsection{Handling Casts Precisely}
The only sensible improvement that was applied to this analysis is the precise
handling of casts. Previously, casting an object would generate a type
constraint corresponding to the type used in the cast. This follows the type
rules of Scala. While valid, it is not precise enough. We instead
re-implemented it using type intersection. Type intersection between two sets
of type constraints $C_1$ and $C_2$ is defined by:

\begin{eqnarray*}
        C_1 \cap C_2 := \bigcup_{(c_1,c_2) ~ \in ~ Cs_1 \times Cs_2} \{ T ~|~ T \models c_1 \land T \models c_2  \}
\end{eqnarray*}

Type intersection allows us to model precisely the possible types resulting
from a type cast. We illustrate this with an example:


\begin{lstlisting}
    trait A
    trait B
    class T1 extends A with B
    class T2 extends B

    val a: A = ...

     ...

    if (...) {
        val b: B = a.asInstanceOf[B]

         ...

    }
\end{lstlisting}


While the static type of \verb=b= is \verb=B=, type intersection gives us a
more detailled description of objects stored in \verb=b=. In fact, only objects that
are both of type \verb=A= and \verb=B= are valid candidates. In the example,
the result of type intersection gives us the constraint $\_ \subtypeeq T1$,
which allows only $T1$ in our case. The constraint obtained from the static
type, $\_ \subtypeeq B$, would allow for $T2$ as well which is not possible in
our case.

\subsection{Evaluation}
It is not obvious whether this simple analysis will give interesting results.
The main purpose being to compute the call-graph, we focus on the improvements
obtained for method calls. That is, for each method call, we compare the number
of targets that we obtain with and without refined type constraints. We use the
official Scala library as benchmark. It is our belief that the library is
representative of typical Scala applications for the purpose of this analysis.

The Scala library contains 122'980 method calls at the compiler phase we
operate. However, only 18'611 of them have a non-unique \emph{static} target.
We represent the improvements provided by our analysis using the scatter plot
in Figure~\ref{fig:scatter}. We can see that the improvements are often
important. Also, we notice that this analysis is at least as precise as the
static type is: it is only improving the set of targets. In overall we obtain
1'278'384 edges in our refined call graph, instead of 3'554'422 without the
analysis, which represents a 64\% reduction.

\begin{figure}[h]
    \begin{center}
    \includegraphics[scale=0.6]{images/scatter}
    \end{center}
    \caption{Improvements due to Type Analysis. Each point represent targets
    computed for one method call. Points on the diagonal represent calls
    without improvements. Points on the Y axis represent calls where the
    analysis reduced the number of targets to a single one.}
    \label{fig:scatter}
\end{figure}

\section{Pointer and Effect Analysis}
\subsection{Overview}
The problem of analyzing pointers is closely related to the field of effects
analysis. Indeed, establishing the relationships between pointers require
understanding how and to what values fields are written to and vice-versa.
Because of this strong inter-dependence, it is profitable to perform both
analyses simultaneously.

Our analysis builds summaries of methods, both in terms of their effects and in
terms of the shape of the heap.

\subsubsection{Flexible Summaries}
Previously, the analysis computed one single, graph-based representation of the
effects of a function. This requires that all method calls have been resolved
and their effects have been merged in. This approach failed to precisely handle
code that highly depends on dynamic dispatch, such as code relying on higher
order functions (HOF).

The main problem when analyzing higher-order functions is that the effect of a
procedure taking a function as argument is directly dependant the actual
argument value. This would require a strongly context-sensitive analysis, which
is usually not compatible with a modular and compositional analysis. It is our
belief that the only way to handle such cases presicely and modularly is to
allow for unresolved method calls to be part of our effect summaries.

Given our previous summaries it is not clear how to include unresolved method
calls in a sound fashion. Instead, we decided to start from CFGs and replace
statements that are precise enough with partial effect "statements". We thus
obtain modified CFGs as effect summaries, granting us a lot of flexibility in
terms of precision: imprecise method calls remain as-is, and their surroundings
that are precise enough can get summarized and grouped into partial effects.

It is important to understand that while we modified our summaries, the
abstract domain and the lattice used in the AI-based analysis remains simple
effect graphs.

\subsubsection{Difficulties of CFG-based Summaries}

The use of CFGs as compositional summaries has a few caveats that needed to be
addressed:

\paragraph{Fix-point Algorithm} When encountering a method call during our
dataflow analysis, we potentially need to inline a \emph{non-flat} effect.
We say that an effect summary is \emph{flat} if it contains only one single
effect statement, in other words it is fully reduced. In the case of a non-flat
effect, we modify the CFG being analyzed and the fix-point algorithm needs to
take this into account. A trivial way of handling such modifications is to
discard all learned facts and restart the analysis from scratch with the new
CFG. We however implemented partial restart, which only restarts the fix-point
computation on the corresponding strongly connected component of the CFG.

\paragraph{Termination Argument} The usual arguments necessary to prove the
termination of the fix-point algorithm are not sufficient in our case. We need
to prove that not only the CFGs are modified in a monotonous fashion, but also
that they themselves reach a fixpoint. In order to make sure that the CFG does
not keep changing, we need to prevent the inlining of recursive or mutualy
recursive functions. The monotoneous modifications are implied given the
CFG inlining we perform: we only add the target CFG alongside the call
statement, that remains here but record the inlined symbol to avoid repeated
inlinings.

\paragraph{Variable Versionning} Local variables inlined from distinct method
calls need to be disambiguated. We thus had to introduce variable versionning
to separate them appropriately.


\subsubsection{Context-Sensitivity}
In some cases, the static signature of a method is not precise enough to
compute meaningful and reusable flat effects, for that reason, we decided to
implement type-based context-sensitivity in our analysis. During the analysis
of a method call, some heuristic compares the call type signature (the
collected type constraints on the receiver and arguments) with the signatures
for which the result is already available. It can then reuse an existing but
more generic summary or request the target method to be reanalyzed given the new
type constraints. Currently, we re-analyze for any new signature, but this could
be refined in the future for better performance, see
\ref{refinedcontextsensitivity}.

\subsection{Effect Summaries}
As mentionned earlier, the summaries that our analysis now computes is a CFG in
which some statements are \emph{effect} statements. Those partial effects
stored in CFG statements are represented as directed graphs. For further
references we now give a breif description of the graphs:

\subsubsection{Nodes}
The nodes in our graphs represent either objects or literal values. We note
that all nodes have associated type constraints describing the types of the
node. We define the following nodes:

\paragraph{INodes:} represent allocated objects. Along with the point at which
they were allocated, the nodes also contain a flag indicating whether the node
represents a single object or not (e.g. if the allocation was done in a loop).

\paragraph{OBNodes:} represent singleton objects. These special nodes are very
similar to INodes, only that it represents globally singleton objects,
constructed using the \verb=object= Scala construct.

\paragraph{LNodes:} represent unknown values for object fields. Load nodes are
typically created when accessing the value of a field of an object that is not
yet known. This can either occur when a field is read, or when it is
weak-updated and we thus need to keep track of its old value. Load nodes get
resolved whenever possible, when merging graphs.

\paragraph{LVodes:} represent unknown values for local variables. Some local
variables do not have any associated value. This is the case for instance for
the arguments and the receiver. By tracking partial effects, we are also
exposed to local variables for which we don't have any associated nodes yet. We
thus need to create a temporary node representing this value. Like load nodes,
LVNodes are resolved as soon as possible, when merging graphs.

\paragraph{Literal Nodes:} represent non-object literal values. This is used to
track modifications to fields that are of non-object types. The granularity of
those literal nodes depend on their type. For instance, we define different
nodes for each boolean literal, but all integer litterals are represented by one
single node.

\subsubsection{Edges}
In our graphs, edges represent operations on fields, and are labelled by the
field name. We only define two different kinds of edges

\paragraph{Inside Edges:} represent write operations. There can be multiple
edges from the same node via the same field, representing multiple possible
write operations (weak update).

\paragraph{Outside Edges:} provide a path to resolve LNodes. The sole purpose
of outside edges is to provide the resolver with a path to follow to translate
load nodes into concrete nodes. We thus have that every load nodes must be
reachable from a non-load node node via outside edges.

\subsection{Implementation}
The pointer and effect analysis is set up to work in two different modes:
\begin{enumerate}
    \item Precise: reduce if possible, but generally can keep imprecise CFG
    statements around, allows CFG-based inlining.
    \item Blunt: forced to reduce to a single effect, only allows inlining by
    effect.
\end{enumerate}

We first take a look at the global \verb=analyze= method responsible to analyze
a method in the specified mode. It is outlined in
Algorithm~\ref{algo:pt:analyze}. It runs a data-flow analysis on the obtained
CFG.  Once the fix-point is reached, and depending on the results obtained, it
reduces it and stores it. In case the CFG has been completely analyzed (i.e. no
unanalyzed method calls), we always reduce to a simple flat effect, even in
Precise mode. It is worth noting that if the analysis is run in blunt mode, all
calls are analyzed and thus a flat but imprecise effect is always available.

\begin{algorithm}
\caption{Effect/Pointer Analysis}\label{algo:pt:analyze}
\begin{algorithmic}[1]
\Function{analyze}{$meth, mode, argsTypes$}
    \State cfg $\gets$ CFGStore get (meth, argsTypes)
    \State analysis $\gets$ new DataFlowAnalysis(cfg, mode)
    \State (result, facts) $\gets$ analysis.computeFixPoint()
    \State // CFG Might have expanded
    \State newCFG $\gets$ analysis.currentCFG
    \If{newCFG.isFlat}
        \State reducedCFG $\gets$ newCFG
    \Else
        \State reducedCFG $\gets$ buildFlatCFG(meth, facts(newCFG.exit))
    \EndIf
    \State
    \If{CFG partially analyzed}
        \State resultCFG $\gets$ partialReduce(newCFG)
    \Else
        \State resultCFG $\gets$ reduceCFG
    \EndIf
    \State
    \If{mode == Precise}
        \State effectsStore $+=$ (meth, Precise, argsTypes) $\mapsto$ resultCFG
    \EndIf
    \State
    \If{resultCFG.isFlat}
        \State effectsStore $+=$ (meth, Flat, argsTypes) $\mapsto$ resultCFG
    \EndIf
    \State
    \State \Return resultCFG
\EndFunction
\end{algorithmic}
\end{algorithm}

\FloatBarrier

\subsection{Transfer Functions}
The handling of flexible summaries required two major modifications in our
existing transfer functions. First, we had to support effect statements.
Also, the handling of method calls had to be modified to account for CFGs as
summaries.

\subsubsection{Effect Statements}
As mentionned previously, our analysis supports flexible summaries in the form
of CFGs mixing normal statements (code) with effects statements (partial
summaries). The transfer function corresponding to the handling of such effect
statements in outlined in Algorithm~\ref{algo:pt:effects}.

For future references, the \verb=getL= and \verb=setL= methods allow us to
access and modify the nodes pointed to by a local variable. The \verb=getNodes=
method work similarly, with one exception: if the local variable does not exist
yet, it creates a LVNode corresponding to this variable, have the variable
point to this node, and return this node along with the new effect graph
(effect graphs are immutable structures). The \verb=locMap= contains all
mappings from local variables to nodes.

We remind the reader that the abstract domain is effects graphs, which is
exactly what is stored in those effect statements. For this reason, the
transfer function needs to appropriately merge the effect graph into the
current data-flow environment to account for the effect statement. For this,
we prepare a map between LVNodes from the inner graph (nodes representing local
variables that are unknown at this point) to nodes from the outer graphs. This
might resolve LVNodes to concrete nodes, or map them to unresolved nodes such
as LNodes or LVNodes. In case it remains unresolved, we check whether we can improve
the type information of this outer node. We then apply the graph merging
algorithm described in Algorithm~\ref{algo:pt:mergegraphs}. Finally, we
overwrite mappings from local variables to nodes using the one present in the
inner graph. This is done because modifications to local variables reflected by
the effect summaries are always strong updates.

\begin{algorithm}
\caption{Transfer function for effect statements}\label{algo:pt:effects}
\begin{algorithmic}[1]
\Function{TFunEffects}{$env, effects$}
    \State innerG $\gets$ effects.env
    \State outerG $\gets$ env
    \State nodeMap $\gets$ new Map[Node, Set[Node]]()
    \For{innerLVNode $\gets$ innerG.LVNodes}
        \State (outerG, outerNodes) $\gets$ outerG.getNodes(innerLVNode.ref)

        \For{outerNode $\gets$ outerNodes}
            \If{outerNode.types != innerNode.types}
                \State newTypes $\gets$ outerNode.types intersectWith innerNode.types
                \If{outerNode is LNode or LVNode}
                    \State newNode $\gets$ outerNode withTypes newTypes
                    \State outerG $\gets$ outerG.splitNode(outerNode, newNode)
                    \State newOuterNodes $\gets$ newOuterNodes $\cup$ newNode
                \Else
                    \State newOuterNodes $\gets$ newOuterNodes $\cup$ outerNode
                \EndIf
            \Else
                \State newOuterNodes $\gets$ newOuterNodes $\cup$ outerNode
            \EndIf

        \EndFor
        \State nodeMap $\gets$ nodeMap + (innerNode $\mapsto$ newOuterNodes)
    \EndFor
    \State
    \State (outerG, newMap) $\gets$ mergeGraphsWithMap(innerG, outerG, nodeMap, effects.uniqueID)
    \State
    \For{(ref, nodes) $\gets$ innerG.locMap}
        \State outerG.setL(ref, nodes flatMap newMap)
    \EndFor
    \State
    \State \Return outerG
\EndFunction
\end{algorithmic}
\end{algorithm}

\subsubsection{Method Calls}
Depending on the mode we currently operate in, along with other criteria, we
may chose to either analyze a method call by inlining the target CFG, or by
applying its simple effect summary. Algorithm~\ref{algo:pt:methcall} describes
the transfer function for method calls. The decision process is outlined in
Algorithm~\ref{algo:pt:shouldwe}.

\begin{algorithm}
\caption{Transfer function for ret = rec.meth(..args..)}\label{algo:pt:methcall}
\begin{algorithmic}[1]
\Function{TFunCall}{$env, call, excludedSymbols$}
    \State oset $\gets$ typesOf(call.rec)
    \State callArgsTypes $\gets$ oset :: (call.args map typesOf)
    \State targets $\gets$ getMatchingMethods(oset, call.meth)

    \State (shouldWe, how,targetCFGs) $\gets$ shouldWeInlineThis(call.meth, callArgsTypes, targets, excludedSymbols)
    \If{shouldWe}
        \If{$how == Precise$}
            \State \Return inlineByCFGs(env, call, targetCFGs)
        \Else
            \State \Return applyEffects(env, call, targetCFGs)
        \EndIf
    \Else
        \State // Ignore and/or error
        \State \Return env
    \EndIf
\EndFunction
\end{algorithmic}
\end{algorithm}

\begin{algorithm}
\caption{Checks if and how a certain call should be inlined.}\label{algo:pt:shouldwe}
\begin{algorithmic}[1]
\Function{shouldWeInlineThis}{$meth, callArgsTypes, targets, excludedSymbols$}
    \State $targetsToConsider \gets targets -- excludedSymbols$
    \If{$analysisMode == Precise$}
        \If{targets is empty}
            \State \Return (No, \_, \_)
        \ElsIf{targetsToConsider.size $>$ 3}
            \State \Return (No, \_, \_)
        \ElsIf{targetsToConsider exists \{ t $=>$ isRecursive(t) and $\lnot$shouldFlatInline(t, callArgsTypes, targets) \} }
            \State \Return (No, \_, \_)
        \Else
            \State $availableCFGs \gets$ targetsToConsider map \{ t $=>$
            \If{shouldFlatInline(t, callArgsTypes, targets)}
                \State getFlatCFG(t, callArgsTypes)
            \Else
                \State getAnalyzedCFG(t, callArgsTypes)
            \EndIf
            \State \}

            \If{availableCFGs all flat}
                \State \Return (Yes, Blunt, availableCFGs)
            \Else
                \State \Return (Yes, Precise, availableCFGs)
            \EndIf
        \EndIf
    \Else
        \State \Return (Yes, ~Blunt, ~targetsToConsider map \{ t $=>$ getFlatCFG(t , callArgsTypes)\})
    \EndIf
\EndFunction
\end{algorithmic}
\end{algorithm}


\paragraph{Inlining By CFG}
When inlining a method call by CFG we merge the CFG of the called method
(called \emph{inner} CFG) in the CFG of the method being analyzed currently (called
\emph{outer} CFG). Algorithm~\ref{algo:pt:inlinebycfg} describes this procedure that
mainly consists of four steps:

\begin{enumerate}
    \item Remove the original call, and replace it with a similar call
    statement recording the target methods already inlined. This effectively
    excludes the already inlined targets from getting inlined again, for
    example if the method call occurs in a loop. At the same time, if the
    number of targets increases when computing a fix-point in the loop, the
    additional targets will be inlined as well.

    \item Rename arguments of the inner CFG to match variables used to call the
    outer CFG. In case values (and not variables) where used as arguments in
    the call, and hence renaming is not possible, we need to add appropriate
    assign statements prior to renaming.

    \item Rename local variables of the inner CFG to differentiate multiple
    calls. This emulates the fact that local variables are typically allocated
    on the stack and thus unique per call.

    \item Add the edges of the inner CFG in the outer CFG.
\end{enumerate}

\begin{algorithm}
\caption{inlineByCFGs: Inlining a method call by merging its CFG in the current CFG.}\label{algo:pt:inlinebycfg}
\begin{algorithmic}[1]
\Function{inlineByCFGs}{$env, call, targetCFGs$}
    \If{targetCFGs empty}
        \State \Return env
    \Else
        \State (nodeA, \_, nodeB) $\gets$ currentCFG.currentEdge
        \State currentCFG $-=$ currentCFG.currentEdge
        \State currentCFG $+=$ Edge(nodeA, call addExcludes targetCFGs, nodeB)

        \For{targetCFG $\gets$ targetCFGs}
            \State renMap $\gets$ Map[Ref,Ref]()
            \State connectStmts $\gets$ Set[Statement]()
            \State

            \For{(callArg, defArg) $\gets$ call.args zipWith targetCFG.args}
                \If{callArg is Ref}
                    \State renMap $+=$ defArg $\mapsto$ callArg
                \Else
                    \State connectStmts $+=$ Assign(defArg, callArg)
                \EndIf

            \EndFor
            \State
            \State renMap $+=$ targetCFG.receiver $\mapsto$ call.receiver
            \State renMap $+=$ targetCFG.retval   $\mapsto$ call.retval
            \State
            \State renCFG $\gets$ targetCFG renamedWithMap renMap
            \State
            \If{connectStmts is Empty}
                \State currentCFG $+=$ Edge(nodeA, Skip, renCFG.entry)
            \Else
                \For{stmt $\gets$ connectStmts}
                    \State currentCFG $+=$ Edge(nodeA, stmt, renCFG.entry)
                \EndFor
            \EndIf

            \State currentCFG $++=$ renCFG.edges

            \State currentCFG $+=$ Edge(renCFG.exit, Skip, nodeB)
        \EndFor

        \State throw new RestartAnalysisWith(currentCFG)
    \EndIf
\EndFunction
\end{algorithmic}
\end{algorithm}

\FloatBarrier

\paragraph{Applying Effects}

In this situation, we already have flat effect graphs representing the
target methods. We thus only need to apply the effects specified by the target
summaries to our current effect. This operation is very similar to applying an
effect statement occuring in the CFG. In fact, it only differs in the pre- and
post-processing of this effect in order to account for arguments and return
values. This procedure is detailled in Algorithm~\ref{algo:pt:applyeffects}.

The first step is to map variables used at the call site to variables present
in the effect summary. We then map the corresponding nodes from the inner graph
to the outer graph. Later, we merge the inner graph into the outer graph using this
map between nodes (the merging algorithm is outlined in
Algorithm~\ref{algo:pt:mergegraphs}). Finally, we modify the local variable
representing the return value to point to the right nodes returned by the
target method.


\begin{algorithm}
\caption{Applying an effect Graph in a CFG}\label{algo:pt:applyeffects}
\begin{algorithmic}[1]
\Function{applyEffects}{$env, call, targetCFGs$}
    \State envs = Set()
    \For{targetCFG $\gets$ targetCFGs}
        \State innerG $\gets$ targetCFG.getFlatEffect
        \State refMap $\gets$ new Map[Ref, Value]()
        \State refMap $\gets$ refMap ++ (targetCFG.args zip call.args)
        \State refMap $\gets$ refMap + (targetCFG.thisRef $\mapsto$ call.rec )
        \State
        \State outerG $\gets$ env
        \State nodeMap $\gets$ new Map[Node, Set[Node]]()
        \For{(iRef, oRef) $\gets$ refMap}
            \State (outerG, outerNodes) $\gets$ outerG.getNodes(oRef)
            \State innerNodes $\gets$ innerG.getL(iRef)
            \State nodeMap $\gets$ nodeMap ++ (innerNodes $\mapsto$ outerNodes)
        \EndFor
        \State
        \State (outerG, newMap) $\gets$ mergeGraphsWithMap(outerG, innerG, nodeMap, call.uniqueID)
        \State
        \State outerRet $\gets$ innerG.getL(targetCFG.retval) flatMap newMap
        \State outerG $\gets$ outerG.setL(call.ret, outerRet)
        \State
        \State envs $\gets$ envs $\cup$ outerG
    \EndFor

    \State \Return Lattice.join(envs)
\EndFunction
\end{algorithmic}
\end{algorithm}

\FloatBarrier

\subsection{Merging Graphs}
An important and frequent operation in our analysis is the merging of two effect
graphs, it either occur during the analysis of a method call for which we have
a flat effect summary, or when we encounter an effect statement, which is
frequent in the precense of partially reduced CFGs. This operation is described
in the \verb=mergeGraphsWithMap= method, outlined in
Algorithm~\ref{algo:pt:mergegraphs}.

It is decomposed into two major steps, the first step is here to extend the
passed nodeMap with additional mappings. The \verb=inlineINode= function is
responsible of checking whether the INode getting inlined corresponds to a
single object or not considering the new graph. To do so, it combines the call
unique ID as well as the allocation point of the INode to check for uniqueness.
This combination done by \verb=safeAdd= prevents infinite repetitions of the
same unique id.

The second step iteratively resolves load nodes and inline IEdges, until it
reaches a fix-point. During the resolution of one load node, we first try to
establish targets on the outer graph by looking for matching edges. We first
consider IEdges as they represent newer values, if no such edges are found,
we look at OEdges. We may also refine the types of the load/lv nodes used to
reach that load node on the way. If no targets can be found, we have to create
a corresponding load node in the outer graph. The only exception is if the node
from which this load node is created is an inside node. Indeed, it is not
possible for an inside node to have unknown fields. However, the lack of
targets might occur since we may have not yet applied the corresponding IEdges
from the inner graph. We thus ignore this as it will be resolved at the time we
reach a fix-point.

When applying write operations in \verb=applyInnerEdges=, we need to make sure
they can handled as strong updates. For this, we first group write operations
appropriately using the node and field they affect in the outer graph. Now, for
each of the node and field affected in the outer graph, we need to look whether
the corresponding nodes of the inner graph mapped only to that single node. If
not, the corresponding update cannot be strong.

\begin{algorithm}
\caption{Merging two effect graphs (1/3)}\label{algo:pt:mergegraphs}
\begin{algorithmic}[1]
\Function{mergeGraphsWithMap}{$outerG, innerG, nodeMap, uniqueID$}
    \State nodeMap $\gets$ nodeMap ++ literalNodes.map(n $\Rightarrow$ n $\mapsto$ n)
    \State nodeMap $\gets$ nodeMap ++ innerG.OBNodes.map(n $\Rightarrow$ n $\mapsto$ n)
    \State
    \Function{inlineINode}{$iNode$}
        \State newId $\gets$ iNode.allocPoint safeAdd uniqueID
        \State
        \State iNodeUnique    $\gets$ INode(newId, true, iNode.types)
        \State iNodeNotUnique $\gets$ INode(newId, false, iNode.types)
        \If{outerG contains iNodeNotUnique}
            \State \Return iNodeNotUnique
        \ElsIf{outerG contains iNodeUnique}
            \State outerG $\gets$ outerG.replaceNode(iNodeUnique, iNodeNotUnique)
            \State \Return iNodeNotUnique
        \ElsIf{iNode.isSingleton}
            \State outerG $\gets$ outerG.addNode(iNodeUnique)
            \State \Return iNodeUnique
        \Else
            \State outerG $\gets$ outerG.addNode(iNodeNotUnique)
            \State \Return iNodeNotUnique
        \EndIf
    \EndFunction
    \State
    \State nodeMap $\gets$ nodeMap ++ innerG.INodes.map(n $\Rightarrow$ n $\mapsto$ inlineINode(n))
    \algstore{mergegraph}
\end{algorithmic}
\end{algorithm}
\begin{algorithm}
    \ContinuedFloat
\caption{Merging two effect graphs (2/3)}\label{algo:pt:mergegraphs}
\begin{algorithmic}[1]
    \algrestore{mergegraph}
    \Function{resolveLoadNode}{$lNode$}
        \State innerFromNodes $\gets$ innerG.OEdges.filter(\_.to == lNode).map(\_.from)
        \State fromNodes $\gets$ new Set[(Node, Set[Node])]
        \For{ifn $\gets$ innerFromNodes}
            \If{ifn is LNode}
                \State fromNodes $\gets$ fromNodes + (ifn, resolveLoadNode(ifn))
            \Else
                \State fromNodes $\gets$ fromNodes + (ifn, nodeMap(ifn))
            \EndIf
        \EndFor
        \State
        \State results $\gets$ Set[Node]
        \State
        \For{(innerFromNode, tmpNodes) $\gets$ innerFromNodes; tmpNode $\gets$ tmpNodes}
            \State newTypes $\gets$ tmpNodes.types intersectWith innerFromNode.types

            \If{!newTypes.isEmpty}
                \If{tmpNode is LNode or LVNode}
                    \State newNode $\gets$ tmpNode withTypes newTypes
                    \State outerG  $\gets$ outerG.splitNode(tmpNode, newNode)
                    \State node $\gets$ newNode
                \Else
                    \State node $\gets$ tmpNode
                \EndIf

                \State targets $\gets$ outerG.getWriteTargets(node, lNode.field)

                \If{targets is empty}
                    \State targets $\gets$ outerG.getReadTargets(node, lNode,field)
                \EndIf

                \If{targets is empty}
                    \If{node is INode}
                        \State // Ignore for now
                    \Else
                        \State newLNode $\gets$ .. create new LNode ..
                        \State outerG $\gets$ outerG.addNode(newLNode).addOEdge(node, field, newLNode)
                        \State results $\gets$ results $\cup$ \{ newLNode \}
                    \EndIf
                \Else
                    \State results $\gets$ results $\cup$ targets
                \EndIf
            \EndIf
        \EndFor
    \EndFunction
    \algstore{mergegraph2}
\end{algorithmic}
\end{algorithm}
\begin{algorithm}
    \ContinuedFloat
\caption{Merging two effect graphs (3/3)}\label{algo:pt:mergegraphs}
\begin{algorithmic}[1]
    \algrestore{mergegraph2}
    \Function{applyInnerEdges}{$newOuterG, nodeMap$}
        \Repeat
            \State oldNewOuterG  $\gets$ newOuterG
            \State mappedEdges $\gets$ Set[(IEdge, Node)]()
            \For{ie $\gets$ innerG.IEdges; mfrom $\gets$ nodeMap(ie.from); mto $\gets$ nodeMap(ie.to)}
                \State  mappedEdges $\gets$ mappedEdges + (IEdge(mfrom, ie.field, mto), v1)
            \EndFor

            \For{ ((newFrom, field), mEdges) $\gets$ mappedEdges groupBy (edge.from, edge.via) }
                \State (edges, oldFroms) $\gets$ mEdges unzip
                \State allowStrong $\gets$ oldFroms.forAll(nodeMap(\_).size == 1)
                \State
                \State newOuterG $\gets$ newOuterG.write(newFrom, field, edges.map(\_.to), allowStrong)
            \EndFor
        \Until{newOuterG == oldNewOuterG}
    \EndFunction
    \State
    \Repeat
        \State oldNodeMap $\gets$ nodeMap
        \State oldOuterG  $\gets$ outerG
        \For{lNode $\gets$ innerG.loadNodes}
            \State nodeMap $\gets$ nodeMap ++ (lNode $\mapsto$ resolveLoadNode(lNode))
        \EndFor
        \State outerG $\gets$ applyInnerEdges(outerG, nodeMap)
    \Until{oldOuterG == outerG and oldNodeMap == nodeMap}

    \State \Return (outerG, nodeMap)

\EndFunction
\end{algorithmic}
\end{algorithm}

\FloatBarrier

\section{Future Work}
We can identify multiple ideas for further development:

\subsection{Refined Context-Sensitivity}
\label{refinedcontextsensitivity}
Instead of reanalyzing whenever the type signature changes, we could instead
identify arguments that may have a big effect on the precision of the analysis
One possible way to detect such relevant arguments is to see if they are either
involved in imprecise method calls, or passed as relevant arguments to other
methods.

\subsection{Partial Reductions}
Currently, the reductions performed on effects are trivial. In
Algorithm~\ref{algo:pt:analyze}, we see that if any method call has not been
analyzed, it relies on \verb=partialReduce=. Otherwise, it reduces to a
single flat effect. However, the current implementation of \verb=partialReduce=
is the identity function, keeping the CFG as is. Implementing this partial
reduction properly will greatly improve the performance of the analysis, as
currently partially analyzed CFG grow rapidly.

\subsection{Improved Heuristic for Precise v.s. Flat inlining}
Currently, the criteria to inline a method call as a flat effect depends on the
call signature: If all arguments have a single static type, we always inline as
a flat effect. This is however suboptimal for two main reasons:
\begin{enumerate}
    \item We may want to use flat inlining even if not all arguments are not fully
precise, it might be the case if the argument is not relevant for the effect
computation.
    \item The type of the argument being precise is not precise enough, we might
benefit from inlining as a CFG. For instance, if the target method calls a
method on a field of the passed argument.
\end{enumerate}

\newpage
\appendix
\section{Additional Algorithms}

\begin{algorithm}
\caption{Decides whether using flat effects should be sufficiently
precise.}\label{algo:pt:shouldwe}
\begin{algorithmic}[1]
\Function{shouldFlatInline}{$meth, callArgsTypes, targets$}
    \State \Return callArgsTypes forall \{ oset $=>$ oset.resolveTypes.size == 1 \}
\EndFunction
\end{algorithmic}
\end{algorithm}

\begin{algorithm}
\caption{Computes a flat CFG for a certain call signature}\label{algo:pt:getflatcfg}
\begin{algorithmic}[1]
\Function{getFlatCFG}{$meth, callArgsTypes$}
    \If{effectsStore contains (meth, Flat, callArgsTypes)}
        \State \Return effectsStore get (meth, Flat, callArgsTypes)
    \Else
        \State effectsStore += (meth, Flat, callArgsTypes) $\mapsto$ EmptyEffect
        \State effect $\gets$ analyze(meth, Blunt, callArgsTypes)
        \Repeat
            \State effect $\gets$ analyze(meth, Blunt, callArgsTypes)
        \Until{effect no longer changes}

        \State \Return effect
    \EndIf
\EndFunction
\end{algorithmic}
\end{algorithm}

\begin{algorithm}
\caption{Compute an effect CFG for a certain call signature}\label{algo:pt:getanalyzedcfg}
\begin{algorithmic}[1]
\Function{getAnalyzedCFG}{$meth, callArgsTypes$}
    \If{effectsStore contains (meth, Precise, callArgsTypes)}
        \State \Return effectsStore get (meth, Precise, callArgsTypes)
    \Else
        \State \Return analyze(meth, Precise, callArgsTypes)
    \EndIf
\EndFunction
\end{algorithmic}
\end{algorithm}

\end{document}
