% -*- Mode:TeX -*-

%% The documentclass options along with the pagestyle can be used to generate
%% a technical report, a draft copy, or a regular thesis.  You may need to
%% re-specify the pagestyle after you \include  cover.tex.  For more
%% information, see the first few lines of mitthesis.cls.

%\documentclass[12pt,vi,twoside]{mitthesis}
%%
%%  If you want your thesis copyright to you instead of MIT, use the
%%  ``vi'' option, as above.
%%
%\documentclass[12pt,twoside,leftblank]{mitthesis}
%%
%% If you want blank pages before new chapters to be labelled ``This
%% Page Intentionally Left Blank'', use the ``leftblank'' option, as
%% above.

\documentclass[12pt,vi,twoside]{mitthesis}
\usepackage{lgrind}
\pagestyle{plain}

%\usepackage{graphicx,epsfig,amsfonts,amssymb,amsmath,subfigure,setspace,url,wrapfig,caption,boxedminipage,multirow}
\usepackage{graphicx,epsfig,amsfonts,amssymb,amsmath,url,caption,subfig}

%% This bit allows you to either specify only the files which you wish to
%% process, or `all' to process all files which you \include.
%% Krishna Sethuraman (1990).

%\typein [\files]{Enter file names to process, (chap1,chap2 ...), or `all' to
%process all files:}
%\def\all{all}
%\ifx\files\all \typeout{Including all files.} \else \typeout{Including only \files.} \includeonly{\files} \fi

%\includeonly{chapter5/chapter5}
%\includeonly{appa/appa}

\setlength{\marginparwidth}{1in}
\raggedbottom

% EXTRA CODE: remove this section in final copy
%\usepackage{ifpdf,showkeys}
% \usepackage{xcolor}
%\newcommand{\jsec}[1]{\marginpar{\fcolorbox{yellow}{yellow}{\parbox{0.7in}{\raggedright \color{blue} \tiny #1 }}}}
%\newcommand{\hsec}[1]{\vfill \fcolorbox{lime}{lime}{\parbox{5in}{\raggedright \bf \color{black} #1 }}}
%  
\newif\ifPDF
\ifx\pdfoutput\undefined\PDFfalse \else\ifnum\pdfoutput > 0\PDFtrue
\else\PDFfalse \fi \fi
%  
 \ifPDF \usepackage{float}
%   \usepackage[pdftex, pdfstartview={FitH}, pdfpagelayout={TwoColumnLeft},bookmarksopen=false, plainpages = false, colorlinks=true, citecolor = blue, urlcolor = blue, filecolor=blue , pagecolor=red, pagebackref=true, linkcolor=blue, hypertexnames=false, plainpages=false, pdfpagelabels ]{hyperref}

\usepackage[pdftex, pdfstartview={FitH}, pdfpagelayout={TwoColumnLeft},bookmarksopen=false, plainpages = false, colorlinks=false, pagebackref=true, hypertexnames=false, plainpages=false, pdfpagelabels ]{hyperref}
 \fi
% end of EXTRA CODE

\usepackage[numbers,sort&compress]{natbib}
\usepackage{mathpazo}

\usepackage{listings}
\usepackage{placeins}
\usepackage{algorithm}
\usepackage{algorithmicx}
\usepackage{algpseudocode}

% "define" Scala
\lstdefinelanguage{scala}{
  alsoletter={@,=,>},
  morekeywords={abstract, case, catch, class, def, do, else, extends, false, final, finally, for, if, implicit, import, match, new, null, object, 
override, package, private, protected, requires, return, sealed, super, this, throw, trait, try, true, type, val, var, while, with, yield, domain, 
postcondition, precondition,invariant, constraint, assert, forAll, in, _, return, @generator, ensure, require, holds, ensuring,=>},
  sensitive=true,
  morecomment=[l]{//},
  morecomment=[s]{/*}{*/},
  morestring=[b]"
}

\newcommand{\codestyle}{\small\sffamily}

\lstset{
%  frame=tb,
  language=scala,
%  aboveskip=3mm,
%  belowskip=3mm,
%  lineskip=-0.1em,
  showstringspaces=false,
  columns=fullflexible,
  mathescape=true,
  numbers=none,
  numberstyle=\tiny,
  basicstyle=\codestyle
} 
\newcommand{\subtypeeq}{\sqsubseteq}
\newcommand{\subtype}{\sqsubset}

% PS: extra macros, definitions, etc.
%\usepackage{algorithm}
%\usepackage{algorithmicx}
%\usepackage{algpseudocode}
%\usepackage{multirow}

\include{defs}

\begin{document}
 

% -*-latex-*-
% $Log: cover.tex,v $
% Revision 1.7  2001/02/08 18:53:16  boojum
% changed some \newpages to \cleardoublepages
%
% Revision 1.6  1999/10/21 14:49:31  boojum
% changed comment referring to documentstyle
%
% Revision 1.5  1999/10/21 14:39:04  boojum
% *** empty log message ***
%
% Revision 1.4  1997/04/18  17:54:10  othomas
% added page numbers on abstract and cover, and made 1 abstract
% page the default rather than 2.  (anne hunter tells me this
% is the new institute standard.)
%
% Revision 1.4  1997/04/18  17:54:10  othomas
% added page numbers on abstract and cover, and made 1 abstract
% page the default rather than 2.  (anne hunter tells me this
% is the new institute standard.)
%
% Revision 1.3  93/05/17  17:06:29  starflt
% Added acknowledgments section (suggested by tompalka)
%
% Revision 1.2  92/04/22  13:13:13  epeisach
% Fixes for 1991 course 6 requirements
% Phrase "and to grant others the right to do so" has been added to
% permission clause
% Second copy of abstract is not counted as separate pages so numbering works
% out
%
% Revision 1.1  92/04/22  13:08:20  epeisach
\title{Toward Interprocedural Effect and Pointer Analysis for Scala}

\author{Etienne Kneuss}

\prevdegrees{
BSc., Computer Science\\
\'{E}cole Polytechnique F\'{e}d\'{e}rale de Lausanne (2009) }

\department{School of Computer and Communication Sciences}
% If the thesis is for two degrees simultaneously, list them both
% separated by \and like this:
% \degree{Doctor of Philosophy \and Master of Science}
\degree{Master of Science in Computer Science}
\degreemonth{August}
\degreeyear{2011}
\thesisdate{June 24, 2011}

%% By default, the thesis will be copyrighted to MIT.  If you need to copyright
%% the thesis to yourself, just specify the `vi' documentclass option.  If for
%% some reason you want to exactly specify the copyright notice text, you can
%% use the \copyrightnoticetext command.
%\copyrightnoticetext{\copyright IBM, 1990.  Do not open till Xmas.}

% If there is more than one supervisor, use the \supervisor command
% once for each.
\supervisor{Viktor Kuncak}{Professor}

% This is the department committee chairman, not the thesis committee
% chairman.  You should replace this with your Department's Committee
% Chairman.
\chairman{Prof. XXX}{Associate Department Head\\Chair,
Committee on Graduate Students}

% Make the titlepage based on the above information.  If you need
% something special and can't use the standard form, you can specify
% the exact text of the titlepage yourself.  Put it in a titlepage
% environment and leave blank lines where you want vertical space.
% The spaces will be adjusted to fill the entire page.  The dotted
% lines for the signatures are made with the \signature command.
\maketitle

% The abstractpage environment sets up everything on the page except
% the text itself.  The title and other header material are put at the
% top of the page, and the supervisors are listed at the bottom.  A
% new page is begun both before and after.  Of course, an abstract may
% be more than one page itself.  If you need more control over the
% format of the page, you can use the abstract environment, which puts
% the word "Abstract" at the beginning and single spaces its text.

%% You can either \input (*not* \include) your abstract file, or you can put
%% the text of the abstract directly between the \begin{abstractpage} and
%% \end{abstractpage} commands.

% First copy: start a new page, and save the page number.
\cleardoublepage
% Uncomment the next line if you do NOT want a page number on your
% abstract and acknowledgments pages.
% \pagestyle{empty}
\setcounter{savepage}{\thepage}
\begin{abstractpage}
Static program analysis techniques working on object-oriented languages require
precise knowledge of the aliasing relation between variables. This knowledge is
important to, among other things, understand the read and write effects of
method calls on objects. Understanding such effects in turn enables compiler
optimizations and other code transformations such as automated parallelization.
This thesis presents a combination of a pointer analysis with a memory effect
analysis for the Scala programming language. Our analysis is based on abstract
interpretation, and computes summaries of method effects as graphs. This
representation allows the analysis to be compositional. Our second contribution
is an implementation of our analysis in a tool called \insane. Our tool is
built as a plugin for the official Scala compiler. It accepts any Scala
program, and is freely available.

\end{abstractpage}

% Additional copy: start a new page, and reset the page number.  This way,
% the second copy of the abstract is not counted as separate pages.
% Uncomment the next 6 lines if you need two copies of the abstract
% page.
% \setcounter{page}{\thesavepage}
% \begin{abstractpage}
% Static program analysis techniques working on object-oriented languages require
precise knowledge of the aliasing relation between variables. This knowledge is
important to, among other things, understand the read and write effects of
method calls on objects. Understanding such effects in turn enables compiler
optimizations and other code transformations such as automated parallelization.
This thesis presents a combination of a pointer analysis with a memory effect
analysis for the Scala programming language. Our analysis is based on abstract
interpretation, and computes summaries of method effects as graphs. This
representation allows the analysis to be compositional. Our second contribution
is an implementation of our analysis in a tool called \insane. Our tool is
built as a plugin for the official Scala compiler. It accepts any Scala
program, and is freely available.

% \end{abstractpage}

\cleardoublepage

\section*{Acknowledgments}
First and foremost I would like to thank my supervisor, Viktor Kuncak, for his
sustained enthousiasm, inspired suggestions and exemplary guidance thourough
the course of this thesis. I would also like to thank him for giving me the
opportunity to work on such interesting subjects, and look forward to working
with him as I continue with my research. I would also like to thank Philippe
Suter for his continuous feedback and guidance without which this thesis would
not have been possible.

I would like to extend my sincere thanks to the members of the Laboratory for
Automated Reasoning and Analysis for numerous stimulating discussions: Ali,
Andrej, Eva, Giuliano, Hossein, Ruzica, Swen and Tihomir: thanks. I would also
like to thank Lukas Rytz for his patience and technical support regarding the
Scala compiler. Finally I thank those who continuously supported me; my
family, my friends, and of course Aline.




%%%%%%%%%%%%%%%%%%%%%%%%%%%%%%%%%%%%%%%%%%%%%%%%%%%%%%%%%%%%%%%%%%%%%%
% -*-latex-*-

\pagestyle{plain}
\include{contents}

\chapter{Introduction} \label{chap:intro}
Pointer analysis is a static analysis technique that builds information on the
relations between pointers and allocated objects. It is also often referred to
as points-to or alias analysis.
In object-oriented languages such as Scala, the use of pointers is
pervasive, rendering even basic static analyses techniques brittle. It is
thus often necessary to establish information on the aliasing relations between
variables, as well as some knowledge of the shape of structures stored on the
heap. This then enables opportunities to run more analyses or to perform
compiler optimizations.

Effect analyses attempt to summarize the side effects of procedures in a certain
domain. In this thesis, we focus on memory-based effects, and are thus interested
in computing a summary of read and write operations performed on object fields.
Clearly, any such effect analysis needs to rely on a good pointer analysis,
and vice-versa. For this reason, we perform both analyses side by side.

The summary of effects coupled with aliasing information can later be used to
perform various kinds of optimizations or enable more sophisticated analyses.
For instance, if we establish that two sequential operations affect disjoint
parts of the heap, we could safely run them in parallel. Also, given a precise
alias information, we could perform some form of typestate analysis, which
consists in checking that objects are used following a certain protocol. A
typical example is objects representing files: it is required that you first
open a file before reading from it. Such
analyses\cite{DBLP:journals/tosem/FinkYDRG08} require a precise alias analysis
to limit the amount of spurious warnings.

Our analysis is based on abstract interpretation
\cite{DBLP:conf/popl/CousotC77,DBLP:conf/popl/CousotC02}. The abstract
representation consists of graphs and is closely based on \cite{Salcianu2006}.
Such graphs are built so that the analysis is compositional. One of the
challenge of such analysis is to provide a representation for the effects of
a method such that it adapts well to the various calling context. For the
analysis to be compositional, we cannot by design rely on much information from
specific call sites.

\section{Contributions}
This thesis makes the following contributions:
\begin{itemize}
    \item
    We present an inter-procedural effect and alias analysis for the Scala
programing language. Our analysis works on arbitrary Scala code, assuming the
absence of concurrency and provided that the complete source code is available.
Our analysis builds on previous work on inter-procedural pointer analysis for
Java \cite{Salcianu2006}. 
We adapted and extended the precision and scope of the original technique with the following features:
    \begin{itemize}
        \item A differentiation between strong and weak field updates to detect
definitely destructive assignments.
        \item A refinement of the allocation
site abstraction that summarizes sets of object;
by incorporating part of the call-stack information in the labelling of
allocation sites, we increase the precision of the analysis for some common
patterns, such as factory methods.
        \item A recency abstraction to be able to determine when an allocation
site abstracts a unique object (singletons).
    \end{itemize}

    \item We have implemented our analysis in a tool called {\insane} whose
source code is freely
available.\footnote{\url{http://github.com/colder/insane}} {\insane} extends
the official Scala compiler and can thus in principle be used with any Scala
program.
% Notable features include:
%     \begin{itemize}
%         \item A backend storage system for intermediate graph results using a
%         database.
%         \item A simple way to describe the effects of unanalyzable methods, such
%         as the ones from the Java library.
%     \end{itemize}
\end{itemize}

\section{Outline}
The rest of this thesis is organized as follows: Chapter~\ref{chap:overview}
gives a quick overview of the tool, followed by an in-depth description of the
initial analysis phases. Chapter~\ref{chap:pointer} describes in full details
the pointer and effect analysis phase. In Chapter~\ref{chap:implementation},
we describe implementation details. Chapter~\ref{chap:related} describes
previous work done in the field of pointer and effect analysis. We then
conclude in Chapter~\ref{chap:conclusion} with some ideas for future work.

\section{Tool Overview}
The tool is made of five distinct phases that are topologically ordered in terms
of requirements. We briefly describe each of the phases necessary for the
overall analysis. Every major phases will be later described in full details in
their respective section.

The tool is decomposed as follows, in order:
\begin{enumerate}
    \item{Function Extraction}: extracts function definitions seen in the code, as well as
    assertions or requirements.

    \item{CFG Generation}: generates a control flow graph for each function
    definition, composed of simple assign statements.

    \item{Class Hierarchy}: collects information in order to find all subclasses of a class.

    \item{Type Analysis}: analyze potential runtime types and builds initial callgraph.

    \item{Pointer Analysis}: analyze aliasing effects occurring in each function.
\end{enumerate}

\section{Function Extraction and CFG Generation}

Each function definition seen in the code is first collected. We start by
extracting invariants as well as pre- and post- conditions explicitly stated in
the code. We illustrate how Scala provides a way to express them in
Figure~\ref{fig:fe:example1}.

\begin{figure}[h]
    \centering
\begin{lstlisting}
class A {
  def test(a: A, b: A) = {
    require(a ne b) // pre-condition

    var c = a
    while(c ne b) {
        assert(c ne null) // invariant
        c = c.next
    }
    c

  } ensuring( r => r eq b ) // post-condition
}
\end{lstlisting}
    \caption{Expressing invariants and pre-/post-conditions in Scala}
    \label{fig:fe:example1}
\end{figure}

We then generate the control flow graph of each function definition, by converting
complex statements into simple assignments. We define the set of
simple assignments in Figure~\ref{fig:cfg:statements}.

\FloatBarrier
\begin{figure}[h]
    \centering

    \begin{tabular}{ l | l }
        CFG Statement               & Code example \\
        \hline
        AssignCast       & \verb/r = v.asInstanceOf[T]/  \\
        AssignTypeCheck  & \verb/r = v.isInstanceOf[T]/  \\
        AssignVal        & \verb/r = v/  \\
        AssignFieldRead  & \verb/r = obj.f/  \\
        AssignFieldWrite & \verb/obj.f = v/  \\
        AssignNew        & \verb/r = new T/  \\
        AssignApplyMeth  & \verb/r = obj.meth(..args..)/  \\
        AssignEQ         & \verb/r = v1 eq v2/  \\
        AssignNE         & \verb/r = v1 ne v2/  \\
    \end{tabular}

    \caption{CFG Statements}
    \label{fig:cfg:statements}
\end{figure}

It is worth noting that Scala converts most operations to method calls, and
introduce implicit getters and setters for non-private fields.  For example,
the expression 
\begin{lstlisting}
val a = 2 * this.f
\end{lstlisting}
will be translated by the compiler into 
\begin{lstlisting}
val a = 2.*(this.f())
\end{lstlisting}
where \verb/*/ is a method on the class Int, and \verb/f/ is the implicit
getter method. To illustrate the CFG generation phase, we provide in
Figure~\ref{fig:cfg:example1} the graph for the method defined in
Figure~\ref{fig:fe:example1}.

Since we remove the pre- and post-conditions from the actual code before
generating its control flow graph, we therefore assume that it is pure.

\section{Type Analysis}

\subsection{Introduction}
Object oriented languages such as Scala implement a feature called
\emph{dynamic dispatch}: the target of a method call is only determined at
runtime, based on the actual runtime type of the receiver. This feature is
essential in object oriented languages as it allows polymorphism. Consider the
Scala code in Figure~\ref{fig:ta:example1}: the compile type of \verb/obj/ in
\verb/A.test/ is \verb/A/, but the target of the method call could either be
\verb/A.foo/ or \verb/B.foo/, based on the actual type of the value of
\verb/obj/, which is only fully determined at runtime.

\begin{figure}[h]
    \centering
\begin{lstlisting}
class A {
  def test(obj: A) {
    obj.foo()
  }
  def foo() {
    println("A")
  }
}

class B extends A {
  override def foo() {
    println("B")
  }
}
\end{lstlisting}
    \caption{Dynamic dispatch}
    \label{fig:ta:example1}
\end{figure}

In general, every redefinitions of \emph{foo} in all subclasses of \emph{A}
could be targets of this method call. We formalize this concept by associating
for each method calls a set of targets $CT$. For this example, we have:

\begin{eqnarray*}
    CT(\verb/obj.foo()/@p) = \{A.foo, B.foo\}
\end{eqnarray*}
where $p$ is the program point--or label uniquely identifying the call.

Type analysis is responsible to compute this set of targets $CT$. For this
analysis to be valid, the set of targets should include all methods that could
be called at runtime. It may however be imprecise and include methods that will
never be called at runtime.

A simplistic implementation of this analysis would be to consider, given
the call \verb/rec.foo()/ where the receiver is of type \emph{T}, all subtypes
of \emph{T} where method \verb/foo()/ is redefined:

\begin{eqnarray*}
        SimpleCT(\verb/rec.foo(..)/@p) := \{C.foo ~ &|& C \in Classes \land \\
        && C \subtype type(\verb/rec/) \land \\
        && \verb/foo/ \in methods(C) \}
\end{eqnarray*}

where $methods(C)$ is the set of methods explicitely (re)defined in class $C$.
This analysis will be guaranteed to be valid given that the program typechecks,
but it will often be suboptimal, as illustrated in
Figure~\ref{fig:ta:example2}: even though the type of \verb/obj/ is \emph{A},
the only possible target of \verb/obj.foo()/ will be \verb/A.foo/.

\begin{figure}[h]
    \centering

\begin{lstlisting}
class A {
    def invoke {
        val obj = new A()
        obj.foo()
    }
    def foo() {
        println("A")
    }
}

class B extends A {
    override def foo() {
        println("B")
    }
}
\end{lstlisting}

    \caption{Precise call}
    \label{fig:ta:example2}
\end{figure}

\subsection{Our implementation}
Analyzing the targets of method calls can be reduced to analyzing the runtime
types of variables. Those types then fully determines the targets of the call.
Our analysis will thus analyze types that could occur at runtime, for every
variables present in the code. We distinguish three types of variables:
\begin{enumerate}
    \item \verb/A.f/: Field \verb/f/ of class \verb/A/.
    \item \verb/arg/: Argument \verb/arg/ of the function.
    \item \verb/locVar/: Local variable \verb/locVar/.
\end{enumerate}

For each of these variable occurrences in the code, our analysis will compute
the set of runtime types, that we will call \emph{ComputedTypes}, as opposed to
\emph{RuntimeTypes} which is the set of all types that could occur in runtime.
For the resulting type analysis to be valid, the set of computed object types
should be a superset of the types of values assigned to those variables at
runtime.  We thus have the following validity requirement:

\begin{eqnarray*}
    \forall v \in Variables: RuntimeTypes(v) \subseteq ComputedTypes(v)
\end{eqnarray*}

In order to compute the set of types at runtime, we need to track values
assigned to those variables. By doing that, we are immediately faced with two
non-trivial problems:
\begin{enumerate}
    \item The values of arguments are determined by call-sites, determining
    call-sites of a certain method is analogous to determining call targets,
    which is the purpose of type analysis.

    \item Fields can be assigned from multiple locations, within various
    methods. Again, determining whether those methods are
    called, and in which order, require type analysis.
\end{enumerate}

Both of those problems could be solved using a fix-point mechanism. However, at
the cost of some precision, we decided to fall back to a simple implementation
for both arguments and fields:

\begin{eqnarray*}
    ComputedTypes(\verb/A.f/) &:=& \{ T ~|~ T \subtype type(\verb/A.f/) \} \\
    ComputedTypes(\verb/arg/) &:=& \{ T ~|~ T \subtype type(\verb/arg/) \} \\
\end{eqnarray*}

where $type()$ is the type infered by the compiler.

For local variable, we run an flow-sensitive, context-insensitive, abstract
interpretation-based analysis. This analysis computes, at every program point,
the set of all types assigned to local variables. For this analysis to be
efficient, we represent the types of variables at each program point as a
tupple $(T_{sub}, T_{ex})$ where $T_{sub}$ is the set of types from which we
need to include all subtypes and $T_{ex}$ is the set of exact types. We enforce
the property that $T_{sub} \subseteq T_{ex}$.

We thus have a point-wise lattice \emph{L} over pairs of sets of types. Its
point-wise lowest upper bound operation is naturally defined as:
$$
    (T_{sub_a}, T_{ex_a}) \sqcup (T_{sub_b}, T_{ex_b}) = (T_{sub_a} \cup T_{sub_b}, T_{ex_a} \cup T_{ex_b})
$$
We outline in Figure~\ref{fig:ta:tf}
the transfer function for the most important statements.
\FloatBarrier
\begin{figure}[h]
    \centering

    \begin{tabular}{ l | l }
        Statement                 & Transfer function \\
        \hline
        \verb/r = new A/          & $facts[ r \mapsto (\{\}, \{ A \})]$ \\
        \verb/r = v/              & $facts[ r \mapsto facts(v)]$ \\
        \verb/r = null/           & $facts[ r \mapsto (\{\}, \{\})]$ \\
        \verb/r = A.f/            & $facts[ r \mapsto (\{type(\verb/A.f/)\}, \{type(\verb/A.f/)\}) ]$ \\
        \verb/r = rec.meth(..)/   & $facts[ r \mapsto (\{type(\verb/rec.meth/)\}, \{type(\verb/rec.meth/)\}) ]$ \\
        \verb/A.f = v/            & $facts$ (\emph{ignore}) \\
    \end{tabular}

    \caption{Transfer function excerpt}
    \label{fig:ta:tf}
\end{figure}

It is evident that this analysis terminates, as there is only finite ascending
chains in the lattice since $\mathcal{P}(Classes)$ is finite, and the transfer functions
is trivially monotonic.

When the fix-point is reached, we can derive the set of call targets \emph{CT}
for each method call using $facts$ computed at their program point:
\begin{eqnarray*}
    CT(\verb/rec.meth(..)/ @ p) := \{ T.meth ~|~ T \in resolve(facts@p(\verb/rec/)) \land \verb/meth/ \in methods(T) \}
\end{eqnarray*}
where $resolve$ is computing the entire set of types that the pair represents:

$$
resolve((T_{sub}, T_{ex})) := \{ T ~|~ \exists S \in T_{sub}. T \subtype S\} \cup T_{ex}
$$

This analysis provides us with a relatively precise information on call targets
that will be used to compute the initial callgraph. We argue that the obvious
lack of precision in the presence of fields and arguments is not problematic,
since the callgraph is only used to determine groups of mutually recursive
functions. A lack of precision in this analysis will only result in poorer time
performances, and will not impact the precision of the overall analysis. This
is partly explained by the fact that a more precise type analysis will be
performed during the pointer analysis phase. For this reason, this analysis is
arguably overly precise. However, we have seen that it is fast enough in
practice.

\chapter{Pointer Analysis}
\label{chap:pointer}

\chapter{Implementation}
\label{chap:implementation}
This tool has been implemented on the top of the Scala compiler, and can be
built as a plugin of the compiler. One of the main advantage of building it as
a compiler plugin is that it grants us immediate access to the trees and all
type information that we need. The compiler allows us to plug our tool between
two existing phases. Depending on where the plugin is inserted, it dramatically
changes the aspect of the trees. In this case, we decided to put it late in the
compilation process, so that we would not have to deal with closures, inner
classes, mixins, or genericity. This however comes with some cost: we lost
precision in the presence of parametric types, and the amount of code to
analyze is much bigger.

\section{Class Hierarchy}
The first implementation problem we faced while working with the Scala compiler
is that, it does not provide any way to access subtypes of one symbol, only its
super type. For this reason, we had to traverse every symbols defined in the
classpath in order to reconstruct the entire hierarchy, which allowed us to
compute subtypes. This process of traversing every defined symbols in the class
path is costly (minimum 30'000 symbols given that the Java and Scala library are always
included), and represented most of the analysis time for small examples.

For this reason, we decided to store the class hierarchy of the Scala and Java
libraries into a database. We use a nested
set\footnote{\url{http://dev.mysql.com/tech-resources/articles/hierarchical-data.html}}
representation for our hierarchical data, which allows us to retrieve the entire
set of subtypes in one SQL query efficiently.

\chapter{Related Work}
\label{chap:related}
To our knowledge, we are the first trying to perform such alias analysis for
Scala. Although Scala compiles to Java bytecode, and thus any analyzer working
with bytecode could in principle be used for analyzing Scala, the steps performed during
compilation introduce many artifacts. For that reason, an analysis focused on
Scala will be able to provide much more useful and precise result than one
working with arbitrary Java bytecode.

Given the usefulness of alias analysis, it has been constantly worked on in the
past decades and remains an active area. Most of the time, alias analysis is not
the goal but the mean to achieve a more sophisticated analysis.

The work that is naturally the most related to this thesis is the work done by
Alexandru Salcianu in \cite{Salcianu2001,Salcianu2006}, on which this thesis builds.
They provide a compositional graph based pointer analysis that
focus on establishing escaping information. While we also provide a similar
compositional analysis based on graphs, we assign slightly different semantics
to our graphs to cope with strong updates, that they did not support.  We also
handle program points refinement, which allows us to provide a more precise
analysis in the presence of factory methods.

In \cite{DBLP:conf/oopsla/DilligDA10}, they propose to add invariants to
refine pointer relations in the heap. Our analysis is thus less precise in
that regard, as it is currently not path sensitive at all. In
\cite{DBLP:conf/ecoop/ChalinJ07}, they develop a demand-driven alias analysis.
It is however flow insensitive and thus less precise that what we have here.
However, the fact that it is demand-driven is very interesting.

In
\cite{DBLP:conf/oopsla/DiwanMM96,DBLP:conf/oopsla/BaconS96,DBLP:conf/fossacs/JensenS01},
they describe similar type analysis techniques to compute the call graph in the
presence of dynamic dispatch.

One distinction between our work and \cite{Salcianu2006} is that we
differentiate between weak and strong updates. Researchers have looks at ways
to further refine this paradigm by introducing logical predicates that qualify
the uniqueness of the object summary \cite{DBLP:conf/esop/DilligDA10}; although
the technique appears appealing, we do not expect that it would provide much
benefit in our setting. Indeed, our analysis already treats most updates as
strong, and it is not clear whether parametrizing them with predicates would
allow us to perform a strong update when we previously could not.

Different alias analysis techniques have been explored in order to perform type
state analysis \cite{DBLP:journals/tse/StromY86, DBLP:journals/tse/StromY93}.
In \cite{DBLP:conf/pldi/FahndrichD02}, instead of figuring out heap aliasing,
required for type state checking, they propose a type system extension that
inherently restricts aliasing interferences. In
\cite{DBLP:conf/ecoop/HallerO10}, they design an annotation system to describe
messages passed between concurrent objects so that they can be used
without any risk of race-conditions.

Much work has been done in order to obtain guarantees during object
initialization
\cite{DBLP:conf/popl/QiM09,DBLP:conf/oopsla/FahndrichX07,DBLP:conf/ecoop/ChalinJ07}.
For instance, \cite{DBLP:conf/oopsla/FahndrichX07} proposes type-based
techniques to prevent issues from arising during the initialization phase. In
our analysis, we handle initialization in a straightforward manner by inlining
the graphs from the various constructors. This allows us to detect whether at
some point during or after initialization, an object field retain its default
value (i.e.  null).

Constructing a precise call-graph in the presence of higher order functions has
been known to be a problematic analysis. It has been shown that it is in fact
equivalent to dynamic dispatch analysis \cite{DBLP:conf/pldi/MightSH10}. In
Scala, the link between the two is explicit, since closures get compiled into
classes defining an apply method. We could thus benefit from ideas developed by
\cite{DBLP:conf/pldi/Shivers88} and later refined by others to improve our
dynamic dispatch analysis for calls to closures.

\chapter{Conclusion}
\label{chap:conclusion}

\section{Limitations and Future Work}

\subsection{Higher Order Functions}
The presence of Higher Order Functions (HOF) complicates our analysis and
compromise its precision. In Scala, HOFs are represented as objects, instances
of FunctionX classes where X is the number of arguments the function has. To
illustrate, we consider a use of a HOF:
\begin{lstlisting}
def test() = {
    plop(42, _ + 1)
}
def plop(i: Int, f: Int => Int): Int = f(i)
\end{lstlisting}
We have here a function named \verb/plop/ which applies the function $f$ passed
as second argument to its first argument. In \verb/test/ we call that method
with $42$, and the function incrementing its argument by one. The result of
\verb/test/ should thus be $43$. At our phase, the compiler will already have
translated that code to:
\begin{lstlisting}
def test(): Int = {
    plop(42, (new Test$$anonfun$test$1(): Function1))
}
def plop(i: Int, f: Function1): Int = f.apply$mcII$sp(i);

class Test$$anonfun$test$1 extends
  scala.runtime.AbstractFunction1$mcII$sp with Serializable {

  final def apply(x$1: Int): Int =
    Test$$anonfun$test$1.this.apply$mcII$sp(x$1);

  def apply$mcII$sp(v1: Int): Int =
    v1.+(1);

  final def apply(v1: java.lang.Object): java.lang.Object =
    scala.Int.box(
      Test$$anonfun$test$1.this.apply(scala.Int.unbox(v1))
    );
}

\end{lstlisting}
As we can see, the compiler transformed the type $Int => Int$ into the general
type $Function1$. It also transformed the closure \verb/{ _ + 1 }/ into a class
defining, among other things, a \verb/apply$mcII$sp/ method which is the
specialized name of the method for $Int => Int$. The call to $f$ is transformed
into a method call to that \verb/apply$mcII$sp/.

If we recall how type analysis is performed for arguments of methods, we can
immediately see a problem in the presence of HOFs. Indeed, the method
\verb/plop/ takes an argument \verb/f/ of type \verb/Function1/ which is the
super type of all functions of one argument. The runtime types calculated for
$f$ will thus include \emph{all} closures of one argument, including ones with
incompatible types. As a result, any call using $f$ as a receiver will
potentially target every defined closures.

In order to address that issue, we could implement the following three
techniques:

\subsubsection{Exploring Type History}
The main reason why types are generalized is that our analysis runs
after the \emph{errasure} phase, which is responsible of removing type
information that cannot remain at runtime because of JVM limitations (mostly
generic types). Thankfully, the compiler keeps an history of the types
associated to each symbol. We could thus recollect the type of the arguments
prior to the \emph{errasure} phase, allowing us to limit the targets to methods
of compatible type.

\subsubsection{Selective Analysis Inlining}
Even though the previous technique would help eliminate many spurious targets,
it would remain highly imprecise. Figure~\ref{fig:con:inl} provides an example
illustrating the imprecision. Assuming that the closures defined in
\verb/test1/ and \verb/test2/ are the only instance of \verb/Function1/, the
effects inferred for \verb/plop/ will be the combination of the effects of the
two closures. As a result, we will infer that \verb/test1/ writes to the field
$a$ even though it doesn't.

\begin{figure}[h]
    \centering
\begin{lstlisting}
class Test {
    var a: Int = 42

    def test1() = {
        plop(_ + 1)
    }

    def test2() = {
        plop(x => a += 1; a)
    }

    def plop(f: Int => Int) = f(42)
}
\end{lstlisting}
    \caption{Imprecise effects inference}
    \label{fig:con:inl}
\end{figure}

By selectively inlining the analysis of the method \verb/plop/ in \verb/test1/ and
\verb/test2/, we could refine the type of the argument from $Function1$ to the
exact type of the class generated for each closure. Type analysis would then
naturally infer that the \verb/f.apply$mcII$sp/ call in \verb/plop/ has only
one target.  As a result, the effects of \verb/test1/ and \verb/test2/ would be
inferred with respect to the closure they define and use.

\subsubsection{Graph-based Delaying of Method Calls}
Instead of inlining the entire analysis of a method, we explored ways to
generate graphs in which certain calls would remain unresolved. The main idea
is that in the presence of an imprecise function call, we could replace the
call by a special node indicating a method call, and "wait" until the receiver
gets refined to actually apply the method call. We could thus keep the overall
modularity of our analysis, and decide to delay problematic method calls, which
would include but not be limited to HOFs.

Even though this idea is appealing, it is yet still unclear how to safely
manage those delayed calls, notably with respect to strong updates and load
nodes. Due to time restrictions, we could not go further with the implementation
of this interesting technique.



\bibliographystyle{alpha}
\renewcommand{\bibname}{References} % changes the header; default: Bibliography
\bibliography{insane}

\end{document}
