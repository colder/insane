% $Log: abstract.tex,v $
% Revision 1.1  93/05/14  14:56:25  starflt
% Initial revision
%
% Revision 1.1  90/05/04  10:41:01  lwvanels
% Initial revision
%
%
%% The text of your abstract and nothing else (other than comments) goes here.
%% It will be single-spaced and the rest of the text that is supposed to go on
%% the abstract page will be generated by the abstractpage environment.  This
%% file should be \input (not \include 'd) from cover.tex.

%This thesis presents \nct, an extension of the \dpllt\ scheme
%\cite{ganzinger2004dtf,nieuwenhuis2006ssa} for decision procedures for
%quantifier-free first-order logics. In \dpllt, a general Boolean \dpll\ engine
%is instantiated with a \emphdef{theory solver} for the theory $T$. The \dpll\
%engine is responsible for computing Boolean implications and detecting Boolean
%conflicts, while the theory solver detects implications and conflicts in $T$,
%and the communication between the two parts is done through a standardized
%interface. The Boolean reasoning is done on a set of constraints represented as
%\emphdef{clauses}, meaning that formulas have to be converted to conjunctive
%normal form before they can be processed. The process results in the addition of
%variables and a general loss of structure. \nct\ remove this constraint by
%extending the Boolean engine to support the detection of implications and
%conflicts on non--clausal constraints, using techniques working on
%graphical representations of formulas in negation normal form first described
%in \cite{jain2008,jainThesis}. Conversion to negation normal form preserves the
%size and structure of the input formula and does not introduce new variables.
%
%The above scheme \nct\ has been implemented as a tool called \fstp, where the
%theory $T$ under consideration is the quantifier--free theory of uninterpreted
%function and predicate symbols with equality. We describe our
%implementation and give early experimental results.
